\chapter{Introduction}
\label{chap:introduction}

Android, with over 2 billion devices monthly active (as of May 2017),
is the most used mobile operating system \cite{ref3}. Google choice of
designing an operating system both open source and compatible with
different hardware made it an attractive solution for smartphone
manufacturers \cite{ref6}, contributing in increasing its mobile OS market
share up to 86\% in the third quarter of 2016 \cite{ref7}. One of the key
components of Google's operating system is that it is easy for the
developers to use the platform \cite{ref6}, as a consequence the increasing
number of users is directly proportional to the increasing number of
application programmers. This has led to \num{2800000} apps available and
more than 65 billion cumulative downloads since 2010 \cite{ref8, ref9}.

The growth of this platform did not come without drawbacks. It, in
fact, became the target of malwares with the aim of stealing data or
money from the users. Kaspersky lab \cite{ref2} detected in 2016 over 8
million malicious installation packages, showing that android security
is at risk. In late May 2017, as confirmation of this, a malware,
named \emph{Judy}, was discovered in circa 50 applications in the play store
\cite{ref5}. The apps containing the malicious code have been downloaded
approximately 36 million times and have been responsible for
generating fraudulent clicks to gain money from adverts. Another
example of a recent (April 2017) malware available directly from the
play store is the \emph{Dvmap} one \cite{ref11}. It is a sophisticated trojan that
not only installs its modules into the system but it also injects the
code into the system runtime libraries. \num{50000} times is the estimation
of the malicious program total downloads.

Both Google and the scientific community have acknowledged the
severity of the situation. Google, recently, developed a new system,
dubbed \textsc{Play Protect}, actively scanning Android devices and new
versions of apps uploaded to the play store to perform malware
detection \cite{ref4}. The system is brand new (May 2017), and aims at being a
bigger security framework including safe browsing protection, and at
locating and locking the device in case it has been lost. The
scientific community, on the other hand, is aware of this situation
since the early days of Android. Many tools have been developed for
static and dynamic analysis. The first one is performed by analyzing
the source code or the byte code of an application, the \textsc{Androguard}
framework is an example of a tool performing this \cite{ref12}. The second
one, at the contrary, is performed by analyzing a program when is
running, \textsc{Andrubis}, \textsc{Forsafe}, \textsc{Joe Sandbox Mobile} and \textsc{TraceDroid} are
examples of sandboxes performing dynamic analysis \cite{ref10}.

We present, in this paper, an update of the \textsc{TraceDroid} sandbox in
order to increase the compatibility with android applications
\cite{ref1}. This sandbox is the core of dynamic analysis of a larger
automated framework, dubbed \textsc{TraceDroid Analysis Platform (TAP)},
performing both dynamic and static analysis with the integration of
\textsc{Androguard}. \textsc{TraceDroid} is a sandbox based on Android 2.3.4 that works
by modifying the core files of the Dalvik Virtual Machine to enable
profiling and trace dynamically the execution of an application. The
main goal of the update is to port it to android 4.4: the last android
version to have the Dalvik Virtual Machine as the default runtime
environment.

The paper has the following structure:
\begin{itemize}
    \item Chapter~\ref{chap:background}: explains the main concepts about android
      architecture useful at understanding the design and
      implementation of the system being ported.
    \item Chapter~\ref{chap:scope}: the framework design is described in detail in a
      zoom in fashion.
    \item Chapter~\ref{chap:implementation}: the implementation notes and limitations are
      discussed.
    \item Chapter~\ref{chap:evaluation}: provides an evaluation of the system aimed at
      checking the increase of compatibility with android
      applications.
    \item Chapter~\ref{chap:conclusions}: describes future work related to android 7 for
      dynamic analysis and general concluding notes are considered.
\end{itemize}
