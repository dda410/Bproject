\begin{abstract}
The Android operating system, with 2 billion devices monthly active
and an estimated market share over 80\%, it is growing faster than the
expectations. Together with the growth of the number of devices that
use Android there is an increase of malwares directed to the most used
mobile operating system. In 2016 over 8 billions installations of
malicious packages have been detected.

The cyber security community has been aware of these malwares since
the early days of android. Various frameworks were created to provide
comprehensive analysis of android application to detect suspicious and
malicious code. The \textsc{TraceDroid Analysis Platform} is a scalable and
automated framework providing both static and dynamic analysis by
means of integrating \textsc{Androguard} and an android sandbox.

The android sandbox, of the cited framework, is the engine providing
the dynamic analysis functionalities. It works by modifying a default
Android system image to enable profiling and consequently tracing the
entire flow of an application. Every image has its own level of API,
which determines the compatibility with an android application. The
aim of this research is to update the Android sandbox, namely
\textsc{TraceDroid}, from version 2.3 to version 4.4 to increase the
compatibility rate.
\end{abstract}
