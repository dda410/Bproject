\chapter{Conclusion}
\label{chap:conclusions}

In this paper we presented an update of the \textsc{TraceDroid} dynamic
analysis framework. First we provided an overview of the reasons why
the system was originally created and why it is still relevant
nowadays. We also provided the necessary background information, in chapter~\ref{chap:background}, to
understand how the system is working. Secondly we explained the
\textsc{TraceDroid} design and the porting process implementation, in chapters~\ref{chap:scope} and \ref{chap:implementation}. Lastly we
showed, in chapter~\ref{chap:evaluation}, that the data obtained in the evaluation phase confirm the
success of porting the system from Android 2.3.4 (codename
Gingerbread) to Android 4.4 (codename KitKat).

The project repository \cite{ref15} contains all the technical guides, the
compiled images, and the tests output to reproduce the porting
process, to use the system and test it. The guides, furthermore,
provide information on how to create an Android Virtual Device and
emulate the new system image: this should lead to a straightforward
integration of \textsc{TraceDroid} with \textsc{TAP}.

Android 4.4, the new system version, has still an high market share
quota among the OS versions developed by Google but, nevertheless, it
will decrease in the next years. The future of \textsc{TraceDroid} will be a
complete redesign and reimplementation of the system to adapt it to
the new Android runtime (ART) of the most recent versions. \textsc{Artdroid}
\cite{ref28} already shows that it is possible to run a sandbox for dynamic
analysis in this runtime. Google provides also, in the source code,
some files where to implement the profiling of an application
execution \cite{ref29}.
