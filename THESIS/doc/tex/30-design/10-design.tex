\section{What to collect}
\label{sec:scope_design}

The main goal of the sandbox is to perform dynamic analysis of Android
application and to provide extensive and relevant traces of an
executed application. The output should provide an overview of the app
behavior and its control flows. Cyber security experts can,
afterwards, analyze the obtained results to understand whether the
analyzed code is malicious and, hence, be classified as malware.

Android applications are written in Java programming language and
after been compiled in \texttt{.dex} file format they are executed by the
Dalvik Virtual Machine that interprets the code. This intermediate
runtime layer, described extensively in the previous section, is where
the core of dynamic analysis resides. Here the modified version of the
VM, dubbed \textsc{TraceDroid}, traces all the relevant data of a running
process. The profile obtained should contain specific information like
the names of the method called, the method's class descriptor, the
type and value of parameters and returning values and the timestamps
of the computations.
