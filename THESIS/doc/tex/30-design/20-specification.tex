\section{Specification}
\label{sec:scope_specification}

\textsc{TraceDroid} aims at extending the already default Android profiling
feature in giving more accurate and precise information about methods
execution and process flow. The traces should be as close as possible
to the Java source code of the .dex files running in the DVM. The
following are the main requirements that the dynamic analysis program
should implement:
\begin{itemize*}
    \item A common interface that specifies which app should be
      analyzed at runtime.
    \item A common interface to store the output in files which are
      named in such a way to understand the process and the thread the
      traces are generated from.
    \item Full automation through its interfaces in order to
      be plugged in a bigger framework such as TAP.
    \item Output traces that consist of the following:
      \begin{itemize*}
           \item Timestamps of when each computation takes place.
           \item Called method name.
           \item Called method class descriptor.
           \item Called method return type and modifiers.
           \item Values and types of the parameters and return values.
           \item Thrown exceptions unwinding.
           \item Indentation of the calls in order to make it easier
             to understand the results.
      \end{itemize*}
\end{itemize*}      
