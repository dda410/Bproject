\section{Dalvik Virtual Machine}
\label{sec:bg_dalvik}

Google decided to adopt Java as the programming language to write
Android applications due to its popularity and pervasive nature. The
Java bytecode, produced by the language, is interpreted and executed
by a virtual machine. A Java mobile ecosystem already existed, named
JME, but because of its fragmentation Google has created its own suite
of tools and its own VM, called Dalvik Virtual Machine \cite{ref14}.

Google implemented this VM with the awareness of the hardware
constraints of the devices running their operating
system. Smartphones, in fact, have a limited processor speed, limited
RAM size, no swap space and are battery-powered. The runtime core,
nevertheless, is based on modern OS principles and process isolation
is granted by having each application running in its own instance of
the DVM. This, furthermore, improves also security by virtue of the
virtual machine sandboxed nature.

Android OS developers, at the bytecode level, performed also an
optimization by adopting a different format with the primary goal of
preserving memory. The new format, called DEX (Dalvik EXecutable)
bytecode, differs from the standard bytecode since Java \texttt{.class} files,
obtained after compiling the sources, are converted into a single \texttt{.dex}
file using the \texttt{dx} utility. On one hand the standard \texttt{.class} files
have their private heterogeneous constant pool for storing different
types of constants, on the other hand the \texttt{.dex} files have different type
specific constant pools shared among all the classes of the same
program. This constant part accounts for a big part of the Java class
files, circa 61\%. Space is, hence, saved by having extra pointers
referring to the same constants in the shared constant pools.

One of the fundamental components of the DVM is the \emph{Zygote}
process. This special VM process is started at boot and is the parent
of all the future spawned VM. Once started, it initializes the core
libraries and listens for the request of a new application. In this
case the \texttt{fork()} command is called and a new VM instance is
started. Forking a VM allows to share the core libraries and minimize
the startup time. Each Application, as a consequence, runs in its own
isolated process which has a unique \texttt{uid} (user identifier) that can
only access the set of features, verified during installation,
requested in its manifest.xml file. This design enforces not only
minimized load times, but also increases the whole OS security.

Another difference between the JVM and Google implementation is that
the architecture is not Stack-based but register-based. Standard
virtual machines are based on Stack data structures because of its
simplicity. The DVM, in contrast, that is based on registers, lacks
code density (circa 25\% larger source code) but increases the overall
performance by about 32\%. Keeping in mind all the hardware constraints
of the devices the Android OS is designed for, this is an acceptable
compromise.
