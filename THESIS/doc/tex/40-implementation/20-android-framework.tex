\section{Porting limitations}
\label{sec:porting_limitations}

Features and limitations of our implementation are discussed in this
section. Most of the original code concerning \textsc{TraceDroid}
based on Android 2.3.4 has been successfully adapted to the sources of
Android 4.4. There are, however, some functionalities that could not
be ported due to bugs or unexpected behavior. We show in table 4.1 a
comprehensive list of the main features of the system, the files they
are implemented in and their porting status. The shorthand 'P' for
ported, 'NP' for not ported and 'T' for testing is used. A brief
description of the features listed follows the table.

\begin{table}[!h]
    \caption{Overview of the ported features status}
    \label{tab:porting_limitations_table}
    \normalsize
    \tabcolsep=0.15cm
    \begin{tabularx} \linewidth {X l l}
        \toprule
        \textbf{Feature} & \textbf{File} & \textbf{Status} \\
        \midrule
        uid interface & Init.cpp & P \\
        Distinguish method enter, exit and exception throwing & Profile.cpp & P \\
        Tracing return values when profiling a method exit & Profile.h & P \\
        Tracing exceptions when a thrown statement is caught & Profile.h & P \\
        Compute method calls' timestamps and indentation & Profile.cpp & P \\
        Compute method modifiers & Profile.cpp & P \\
        Compute a method's return type and class descriptor & Profile.cpp & P \\
        Compute string representation of parameters and return value & Profile.cpp & P \\
        Filter calls executing bytecode from a system jar & Profile.cpp & T \\
        Compute string representation of a Java object & Profile.cpp & NP \\
        Creating trace files under /sdcard/ & Profile.cpp & P \\
        Printing traces to dump files & Profile.h & P \\
        \bottomrule
    \end{tabularx}
\end{table}

\paragraph{} ~\\


\paragraph{uid interface} ~\\
Allows to trace a specified app by storing its uid in the
\texttt{/sdcard/uid} file.

\paragraph{Distinguish method enter, exit and exception throwing} ~\\
Makes it possible to produce different traces based on whether a
method entrance, a method exit or an exception thrown is being
profiled. \texttt{dvmMethodTraceAdd()} detects the type of action and calls a
method to handle it. The three methods that can be called are
\texttt{handle\_method}, \texttt{handle\_return}, \texttt{handle\_throws}.

\paragraph{Tracing return values when profiling a method exit} ~\\
The \texttt{TRACE\_METHOD\_EXIT} macro declaration is augmented to hold the
returning value of a method. The value is passed as a parameter from
the files where the macro is present.

\paragraph{Tracing exceptions when a thrown statement is caught} ~\\
The \texttt{TRACE\_METHOD\_UNROLL} macro declaration is augmented to hold the
thrown exception of a method. The value is passed as a parameter from
the Exception.cpp file.

\paragraph{Compute method calls' timestamps and indentation} ~\\
Implemented by \texttt{getWhitespace()} method.

\paragraph{Compute method modifiers} ~\\
Implemented by \texttt{getModifiers()} method.

\paragraph{Compute a method's return type and class descriptor} ~\\
Both the functionalities are implemented by the \texttt{convertDescriptor()} method.

\paragraph{Compute string representation of parameters and return value} ~\\
The string representation of a returning value is calculated by
\texttt{parameterToString()}. Since parameters can be more than one,
the computation of their string representation is implemented by
\texttt{getParameters()} and \texttt{getParameterString()},
respectively returning an array of strings and a singular string
representation of this array.

\paragraph{Filter calls executing bytecode from a system jar} ~\\
This feature, implemented in the \texttt{dvmMethodTraceAdd()} method,
allows to omit the traces regarding methods that execute bytecode from
a system jar where also the caller consists of bytecode from a system
jar. Not tracing these calls speeds up the system and excludes to
profile calls assured not to contain malicious code. This feature has
been implemented in a different way compared to the one present in the
original \textsc{TraceDroid} version to solve its unexpected behaviour.

\paragraph{Compute string representation of a Java object} ~\\
This functionality, originally implemented by the
\texttt{objectToString()} method, consists in getting the string
representation of a parameter or of a returning value if it is a Java
object. The method calls the object's Java class \texttt{toString()}
method in the case there is one. Furthermore, a boolean variable,
named \texttt{inMethodTraceAdd}, is also needed in
\texttt{dvmMethodTraceAdd()} to avoid tracing the \texttt{toString()}
calls.

\paragraph{Creating trace files under \texttt{/sdcard/}} ~\\
Implemented by \texttt{prep\_log()} method. It creates dump files for
the current process for each of its threads. The process id and the
thread id are part of the dump filenames (\texttt{dump.PID.TID}). This
allows to distinguish between the traces of each profiled application
thread.

\paragraph{Printing traces to dump files} ~\\
Implemented by the \texttt{ALOGD\_TRACE} macro. It acquires a lock to
write to the dump files and then it writes the trace logs to the
thread dump file.

\paragraph{} ~\\
As it can be seen in table 4.1 most functionalities have been
successfully ported and adapted to Android 4.4 source code. The
features with the 'T' status have not been included in the stable
version of the system since the code implementing them is still in
testing. They are included in a separate source code branch with a
different compiled system image.

In the stable version of \textsc{TraceDroidv4.4}, hence, the filtering of calls
executing bytecode from a system jar was not included. This results in
larger output traces for the profiled application, increasing
also the computation steps required by the system. The second feature
that has not been ported was about computing the string representation
of the parameters and returning Java objects. It has been substituted
by printing the class descriptor of the object and the keyword
\textit{“Object”} to highlight that the parameter or the returning value is a
Java object created by that class.
