\section{Testing}
\label{sec:eval_benchmarks}

The aim of this phase is to test the implemented system in order to
obtain relevant output that could be later analyzed to understand if
the porting was successful. The first step was to decide which
applications to give as input to the new \textsc{TraceDroid} version. We chose
to use the system with a default Android 4.4 application and with 3
external ones not compatible with \textsc{TraceDroid} based on Android
2.3.4. On one hand, testing a default application gave us the output
to understand if the system was producing the expected traces, on the
other, testing 3 previously-not-compatible applications generated the
output to understand if the compatibility of \textsc{TraceDroid} was increased.

The testing environment was the Android emulator with the compiled
\textsc{TraceDroid} 4.4 system image and the Android 4.4 AVD as input. We chose
the Android browser as the default input application. We stored the
app's uid under the common input interface, described in previous
chapters, and then we started one of the browser's activity (opening a
new blank tab) so that the system could trace its execution. After the
applications terminated its execution the trace could be pulled from
the \texttt{sdcard/} in order to be analyzed.

Aptoide, Firefox and waze were the 3 external applications we decided
to use \cite{ref23, ref24, ref25}. The following are the reasons behind this choice:
\begin{enumerate*}
  \item All the three applications had a \textit{minSdkVersion} value, in their
    manifest.xml decompiled with \textsc{Androguard} \cite{ref12}, greater
    than API level 10, the one used by the previous
    \textsc{TraceDroid} version, and smaller or equal to API level 19,
    the one used by the ported \textsc{TraceDroid} version. This is
    essential to evaluate the compatibility increase of the new system
    image.
  \item They are among the most downloaded Android applications \cite{ref26, ref27}.
\end{enumerate*}
The environment used for testing and the steps were the same used for
the default Android browser except for the need to install the
applications since they were not already present in the emulated AVD.

All the 4 used applications were executed with both the \textsc{TraceDroid}
image producing full traces and the testing one omitting
calls executing bytecode from a system Jar. The whole testing process
is reproducible and the guides on how it was performed with the
resulting output can be found on the project repository \cite{ref15}.
